\beginsong{Kfjetina}[by={Tomáš Klus}]
\transpose{-1}
\num
\[G]{Z lidí} se stávaj \[Hm]dospělí protože \[C]není~-- \[D] dost pochopení
a \[G]měním se taky \[Hm]{já~-- tou} \[C]dobou s ní.\[D]..
\[G]pod muší křídla se s\[Hm]chovává vždycky když \[C]tuší~-- že, \[D]prší na duši
\[G]{z nadzemských} výšek mi \[Hm]zamává a srdce \[C]buší ..\[D].
\fin
\chordsoff
\chor
\chordson
Je \[G]kvě-tin\[Hm]{a  } \[C]  \[D]
a j\[G]e-din\[Hm]{á ~-- co} \[C]věří \[D]..
že \[G]zvěři se dá \[Hm]věřit
že \[C]{na trnitým} \[D]keři se dá \[G]spát \[Hm]{.. } \[C]    \[D]  \[D]
\cl
\num
Uprostřed nebe je díra, a právě víra v ní ji~-- otevírá
poslům z cizích světů, jak střelám~-- kulometu,
podléhám ti \ldots{}
podléhám ti \ldots{}
\fin
\chor
\chordson
\[G]{a ty~-} js\[Hm]{i ta} \[C]{~-- láhe}\[D]{v nedopitá}
\[G]skrytá-\[Hm]{áá, } \[C]{ v krytu} \[D]trilobita..\[D]  \[D]
\cl
\num
Na hrotu pera mám větu,  která~-- zůstane do večera
nevyřčena~--  jinak ztrácim cenu..
nechci být autorem frází, jak voda z prolomených hrází,
unášíš mě, \ldots{}
unášíš mě, \ldots{}
\fin
\num
Na tisím malejch kousků rozložím svět a pak zas poskládám no a ty budeš zpět
já budu moudřejší a ty šťastnější, jen mi věř ..
Na tisím malejch kousků rozložím svět a pak zas poskládám no a ty budeš zpět
já budu moudřejší a ty šťastnější, jen mi věř ..      tečka!
\fin
\endsong


