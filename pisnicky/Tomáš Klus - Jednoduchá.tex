\beginsong{Jednoduchá}[by={Tomáš Klus}]
\emptyv
\cseq{\[G] \[C] \[G] \[C] \[G] \[C] \[D]}\\
\cl
\chordsoff
\num
\chordson
Tady kdysi \[G]žila
\[C]{a nevěděla} \[G]nic. \[C]
Pila jako \[G]pila,
\[C]pořád víc a \[G]víc.\[C]
\fin
\freev
A tam žil on,
obyčejnej kluk.
Silnej jako slon,
hubenej jak luk.
\cl
\freev
Nic velkýho netušili,
ani ona, ani on.
Ale pro sebe tu byli
\chordson
\[C]{a vtip} je v \[D]tom… že:
\cl
\chor
\chordson
\[Em]Utopíme \[C]válku
\[G]{v čajovým} \[D]šálku.
\[Em]Budem si \[C]říkat:
„Mám tě \[D]ráda, mám tě rád!“
\cl
\freev
Až pochopíme,
co netušíme.
Budem si říkat
\chordson
\[D]víc než častokrát:
Mám tě \[G]rád!
\[C]Tak hezky to \[G]zní \[C]
a na každej \[G]pád
\[C]jsme spřízně\[G]ný. \[C]
\cl
\num
A já jsem on a ona jsi ty,
šampión přes city.
Každej něco hledá,
kdo to najde, ten to má.
A ty jsi bledá, ale jsi,
\chordson
ale jsi \[D]{má má} má má má\ldots{}
\fin
\repchorus{\emptyspace}
\emptyv
Mám tě rád, tak hezky to zní!
\cl
\endsong


