\beginsong{Poutník a dívka (Ráchel)}[by={Spirituál kvintet}]
%filter: Mlok
\num
Kr\[A]áčel krajem poutník, šel sám,
kr\[D]áčel krajem poutník, šel s\[A]ám,
kráčel krajem poutník, \[C\shrp{}]kráčel \[F\shrp{}m]sám,
tu potkal \[H]dívku, nesla \[H\hidx{7}]džbán, přistoupil \[E\hidx{7}]{k ní} a pravil:
\fin
\chordsoff
\num
\uv{Ráchel, Ráchel, žízeň mě zmáhá,
Ráchel, Ráchel, žízeň mě zmáhá,
Ráchel, Ráchel, žízeň mě zmáhá,
tak přistup blíže, nehodná, a dej mi pít,} a ona:
\fin
\num
\uv{Kdo jsi, kdo jsi, že mi říkáš jménem,
kdo jsi, kdo jsi, že mi říkáš jménem,
kdo jsi, kdo jsi, že mi říkáš jménem,
já tě vidím poprvé, odkud mě znáš?}
\fin
\num
\uv{Ráchel, Ráchel, znám víc než jméno,
Ráchel, Ráchel, znám víc než jméno,
Ráchel, Ráchel, znám víc než jméno,}
pak se napil, ruku zdvih' a kráčel dál.
\fin
\num
Ten džbán, ten džbán z nepálené hlíny,
ten džbán, ten džbán z nepálené hlíny,
ten džbán, ten džbán z nepálené hlíny
v onu chvíli zazářil kovem ryzím.
\fin
\num
Kráčel krajem poutník, šel sám,
kráčel krajem poutník, šel sám,
kráčel krajem poutník, kráčel sám,
\chordson
ač byl \[H]chudý, nepo\[E]znán, přece byl \[A]král.
\fin
\endsong




