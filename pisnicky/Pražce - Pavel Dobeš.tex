\beginsong{Pražce}[by={Pavel Dobeš}]
%filter: Mlok
\num
\[C]Házím tornu na svý záda, feldflašku a \[G\hidx{7}]sumky
\chordsoff
Navštívím dnes kamaráda z železniční průmky
\fin
\chordsoff
\chorusi
\chordson
\[C]Vždyť je jaro, zapni si kšandy
Pozdravuj \[G\hidx{7}]vlaštovky a muziko, ty \[C]hraj
\cl
\num
Vystupuji z vlaku, který mizí v dálce
Stojím v České Třebové a všude kolem pražce
\fin
\num
Pohostil mě slivovicí, představil mě Mařce
Posadil mě na lavici z dubového pražce
\fin
\num
Provedl mě domem, nikde kousek zdiva
Všude samej pražec, jen Máňa byla živá
\fin
\chorusii
To je to jaro, zapni si kšandy
Pozdravuj vlaštovky, a muziko, ty hraj
\cl
\num
Plakáty nás informují~-- přiď pracovat k dráze
Pakliže ti vyhovují rychlost, šmír a saze
\fin
\num
A jestliže jsi labužník a přes kapsu se praštíš
Upečeš i krávu na železničních pražcích
\fin
\num
A naučíš se skákat tak, jak to umí vrabec
Když na nohu si upustíš železniční pražec
\fin
\num
Když má děvče z Třebové rádo svého chlapce
Posílá mu na vojnu železniční pražce
\fin
\num
A když děti zlobí, tak hned je doma mazec
Mikuláš jim nedonesl ani jeden pražec
\fin
\num
Před děvčaty z Třebové chlubil jsem se silou
a pozvedl jsem pražec, načež odvezli mě s kýlou
\fin
\num
Pamatuji pouze ještě operační sál
Pak praštili mě pražcem a já jsem tvrdě spal
\fin
\chor
A bylo jaro, zapni si kšandy
Lítaly vlaštovky a zelenal se háj
\cl
\endsong



