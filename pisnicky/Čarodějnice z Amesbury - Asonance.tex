\beginsong{Čarodějnice z Amesbury}[by={Asonance}]
%filter: Mlok
\num
Zuzana \[Em]byla dívka, \[D]která žila v \[Em]Amesbury
S jasnýma \[G]očima a \[D]řečmi pánům \[Em]navzdory
Souse\[G]dé o ní \[D]říkali, že \[Em]temná kouzla \[Hm]zná
A \[C]že se lidem \[Hm]vyhýbá a s \[C]ďáblem \[D]pletky \[Em]má
\fin\chordsoff\num
Onoho léta náhle mor dobytek zachvátil
A pověrčivý lid se na pastora obrátil
Že znají tu moc nečistou, jež krávy zabíjí
A odkud ta moc pochází, to každý dobře ví
\fin\num
Tak Zuzanu hned před tribunál předvést nechali
A když ji vedli městem, všichni kolem volali:
\uv{Už konec je s tvým řáděním, už nám neuškodíš,
teď na své cestě poslední do pekla poletíš!}
\fin\num
Dosvědčil jeden sedlák, že zná její umění
Ďábelským kouzlem prý se v netopýra promění
A v noci nad krajinou létává pod černou oblohou
Sedlákům krávy zabíjí tou mocí čarovnou
\fin\num
Jiný zas na kříž přísahal, že její kouzla zná
V noci se v černou kočku mění dívka líbezná
Je třeba jednou provždy ukončit ďábelské řádění
\chordson
A všichni křičeli jako posedlí: \uv{Na šibenici \[A]{s ní}!}
\fin\num
Spektrální důkazy pečlivě byly zváženy
Pak z tribunálu povstal starý soudce vážený
\uv{Je přece v knize psáno: nenecháš čarodějnici žít
a před ďáblovým učením budeš se na pozoru mít!}
\fin\num
Zuzana stála krásná s hlavou hrdě vztyčenou
A její slova zněla klenbou s tichou ozvěnou:
\uv{Pohrdám vámi, neznáte nic nežli samou lež a klam,
\chordson
pro tvrdost vašich srdcí jen, jen pro ni umí\[A]rám!}
\fin\num
Tak vzali Zuzanu na kopec pod šibenici
A všude kolem ní se sběhly davy běsnící
Ona stála bezbranná, však s hlavou vztyčenou
Zemřela tiše samotná pod letní oblohou
\fin
\endsong



