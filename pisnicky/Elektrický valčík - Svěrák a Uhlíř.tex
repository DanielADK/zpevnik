\beginsong{Elektrický valčík}[by={Svěrák a Uhlíř}]
%filter: Mlok
\num
\[Cm]Jednoho letního večera na návsi pod starou \[G\hidx{7}]lípou
Hostinský Antonín Kučera vyvalil soudeček s \[Cm]pípou
Neby\[G\shrp{}]{lo to} posvícení, neby\[Cm]{la to} neděle
V naší \[G\shrp{}]obci mezi kopci plni\[G\hidx{7}]{ly se} korbele
\fin
\chordsoff
\chor
\chordson
\[C]Byl to ten slavný den, kdy k nám byl zaveden \[G\hidx{7}]elek\[G\didx{dim}]trický \[G\hidx{7}]proud
Byl to ten slavný den, kdy k nám byl zaveden \[C]elek\[C\didx{dim}]trický \[C]proud
\[C\hidx{7}]Střída\[F]vý, \[G]střída\[Em]vý, \[Am]silný \[Dm]elek\[G\hidx{7}]trický \[C]proud
\[C\hidx{7}]Střída\[F]vý, \[G]střída\[Em]vý, \[Am]zkrátka \[Dm]elek\[G\hidx{7}]trický \[C]proud
\cl
\recite
 Kdo tu všechno byl?
Okresní a krajský inspektor, hasičský a recitační sbor,
poblíže obecní váhy tříčlenná delegace z Prahy,
zástupci nedaleké posádky pod vedením poručíka Vosátky,
početná družina montérů \emph{(jeden z nich pomýšlel na dceru
sedláka Krušiny)}, dále krojované družiny, alegorické vozy,
italský zmrzlinář Antonio Cosi na motocyklu Indián
a svatý Jan, z kamene vytesán!
\cl
\chor
Byl to ten slavný den\ldots{}
\cl
\num
Na stránkách obecní kroniky ozdobným písmem je psáno
Tento den pro zdejší rolníky znamenal po noci ráno
Budeme žít jako v Praze, všude samé vedení
Jedna fáze, druhá fáze, třetí pěkně vedle ní
\fin
\chor
Byl to ten slavný den\ldots{}
\cl
\recite
 Z projevu inženýra Maliny z Elektrických podniků:
Vážení občané, vzácní hosté, s elektřinou je to prosté.
Od pantáty vedou dráty do žárovky nade vraty,
odtud proud se přelévá do stodoly, do chléva.
Při krátkém spojení dvou drátů dochází k takzvanému zkratu.
Kdo má pojistky námi předepsané, tomu se při zkratu nic nestane.
Kdo si tam nastrká hřebíky, vyhoří a začne od píky.
Do každé rodiny elektrické hodiny!
\cl
\chor
Byl to ten slavný den\ldots{}
\cl
\endsong


