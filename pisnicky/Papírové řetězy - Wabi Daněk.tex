\beginsong{Papírové řetězy}[by={Wabi Daněk}]
\capo{3}
\num
\[A]Slepuje očko do očka, \[Hm]řetězy z papíru
\[C\shrp{}m]{Není to} jen tak z plezíru a v \[E\hidx{7}]domě voní vánočka
\[A]{Co kdyby} právě zrovna teď, \[Hm]pohyby prstů stále stejný
\[C\shrp{}m]{Už ví,} že je to beznadějný, \[E\hidx{7}]{už to} ví, že se nedočká
\fin
\chordsoff
\chor
\chordson
\[D]Řetězy \[E\hidx{7}]papírový \[A]změní se \[F\shrp{}m]{na okovy}
\[D]Nikdo jí \[C\shrp{}m]neodpoví, \[E\hidx{7}]nikdo se nezeptá
\[D]{Za šperky} \[E\hidx{7}]safírový \[A]nekoupí \[F\shrp{}m]{to, co} není
\[D]{Je jako} \[C\shrp{}m]{ve vězení,} \[Dm]vánoční \[A]noc, \[Dm]vánoční \[A]noc
\cl
\num
Pomalu chystá večeři, prostírá po paměti
Venku jsou slyšet hlasy dětí a někdo zvoní u dveří
Běží jak voda v potoce, je to jen pošťák s telegramem
Zarámovaný dveřním rámem přeje veselé Vánoce
\fin
\num
Nemusí ani slova číst, je jí to dávno všechno jasné
Tak zavře dveře, světlo zhasne, ke stolu nese bílý list
Nůžkama proužek po proužku odstřihuje z něj mechanicky
A na řetězech jako vždycky přibývá kroužek po kroužku
\fin
\repchorus{\emptyspace}
\cverse
\chordson
\ldots{} \[Dm]vánoční \[A]noc
\cl
\endsong




