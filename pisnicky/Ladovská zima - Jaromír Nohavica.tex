\beginsong{Ladovská zima}[by={Jaromír Nohavica}]
\capo{5}
\freev
\chordsoff
Za vločkou vločka z oblohy padá,
chvilinku počká a potom taje,
na staré sesli sedí pan Lada,
obrázky kreslí zimního kraje.
\cl
\chordsoff
\chor
Ladovská zima za okny je,
a srdce jímá bílá nostalgie,
ladovská zima, děti a sáně,
a já jdu s nima do chrámu páně.
Bim, bam, bim, bam, bim, bam,
bim, bam, bim, bam, bim.
\cl
\freev
To mokré, bílé svinstvo padá mi za límec.
Už čtvrtý měsíc v jednom kuse furt prosinec.
Večer to odhážu, namažu záda,
ráno se vzbudim a zas kurva padá.
Děcka majú zmrzlé kosti,
sáňkujú už jenom z povinnosti,
mrzne jak sviňa, třicet pod nulů,
auto ani neškytne, hrudky se dělají v Mogulu.
Kolony aut krok, sun, krok,
bo silničáři tak jak každý rok,
jsou překvapeni velice,
že sníh zasypal jim silnice.
Pendolino stojí kdesi ku Polomi,
zamrzli mu všecky CD-Romy.
A policajti? Ti to jistí z dálky,
zalezli do Aralky.
\cl
\repchorus{\emptyspace}
\freev
Na Vysočině zavřeli D-1ku,
kamiony hrály na honičku,
tzv. Rally letní gumy,
v tým kopcu u Meziříčí,
spěchali s melounama a teď jsou… \emph{(však víte kde)}
Na ČT1 studio sníh,
Voldanová sedí na saních.
A v Praze kalamita, jak na Sibiři,
tři centimetry sněhu a u Muzea čtyři.
Jak v dálce vidím zasněžený Říp,
říkám si Praotče Čechu, tys byl ale strašný cyp.
Kdyby jsi popošel ještě o pár kilometrů dále,
tak jsem se teď mohl kdesi v teple v plavkách válet.
Místo toho aby se člověk bál zajít do Tesca na nákupy,
jak jsou tam na tých rovných střechách sněhu kupy.
Do toho všeho jak mám zmrzlý nos aj líca,
tak ještě z rádia provokatér Nohavica.
\cl
\repchorus{\emptyspace}
\endsong


