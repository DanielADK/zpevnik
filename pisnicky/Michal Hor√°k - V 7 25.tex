\beginsong{V 7:25}[by={Michal Horák}]
\chor
\[Am]{V sedm} dvacet pět nastupuje do šestnáctky,
\[G]{v sedm} dvacet pět,   r\[E]áno v sedm dvacet pět ji
\[Am]vídám s batohem, mlčky pak jí dávám sbohem,
\[G]protože vždy oněmím,
\[E]oněmím vždy v sedm dvacet -
\cl
\chordsoff
\freev
\chordson
\[AmG]{ná n}\[E]á \[AmG]{na na} ná\ldots{}
\[E]oněmím vždy v sedm dvacet pět.
\cl
\emptyv
\chordson
\cseq{\[Am] \[C] \[G] \[D]}\\
\cl
\num
\chordson
\[Am]{Při ranním} vstávání
\[C]vždy mě raní očekávání \[G]
\chordsoff
přecpané šestnáctky,
\chordson
\[D]přelidněné šestnáctky, však
\[Am]{v době} poslední
\[C]ranní jízdy mi nezevšední,
\[G]ptáte-li se „cože?“
\[E]Tak je to protože –
\fin
\repchorus{\emptyspace}
\num
Kdyby mi tréma ta
nesebrala všechna témata,
jak řeč zavésti
beze trapných předzvěstí.
Až bude zatáčka
nenápadně k ní se namačkám,
pak se jí omluvím,
aspoň tak s ní promluvím, když –
\fin
\repchorus{\emptyspace}
\emptyv
\chordson
\cseq{\[AmF] \[D] \[F]}\\
\cl
\averse
\chordson
\[Am]Říkáte si, asi jak
\[F]zafungoval tenhle naviják,
\[D]řeknu to asi tak –
\[F]nezafungoval nijak.
\[Am]Vážení, jsem nesmělý,
\[F]snažení je zase ztraceno,
\[D]však dřív či později
\[F]zeptám se jí na jméno! \emph{(A pak si ji přidám na fejsbuku.)}
\cl
\emptyv
\chordson
\cseq{\[R.]}\\
\cl
\endsong


