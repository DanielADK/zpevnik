\beginsong{Hej teto!}[by={Michal Horák}]
\begin{multicols}{2}
\num
Měl jsem  \[Gm]{ hlídat}
tetě  \[B] křečka,
\[F]{  dala} mi ho se slovy, ať
\[C]{  u} mě chvíli přečká.
\fin
\chordsoff
\freev
Ale   křečka
uhlí dati
nelze jen tak lehce, obzvlášť
když se vám moc nechce.
\cl
\freev
Po po koji
ať si   běhá,
na gauč jsem si lehnul,
vzápětí se ani nehnul.
\cl
\freev
Tlačilo mě
do zad kromě
polštářů i cosi,
už se vidím tetu prosit:
\cl
\chor
\chordson
Hej \[E\flt{}]  teto,
já ho  \[B] zabil,
\[D]{už se} to nestane, jo
\[Gm]{ten už} asi nevstane.
\cl
\freev
Hej   teto,
nekřič   na mě,
já ho prostě neviděl,
už dost jsem se nastyděl,
hej   teto!
\cl
\emptyv
\chordson
\cseq{\[B] \[D] \[G] \[E] \[B] \[D] \[D\hidx{7}]}\\
\cl
\num
Tak jsem kouknul,
co mě tlačí,
seděl jsem jenom na televizním ovladači,
\fin
\freev
ale křeččí
švitoření
slyšet není,
jenom zvláštní smrad se line od topení.
\cl
\freev
Zase se mi
dech zatajil,
spečeného křečka bych před tetou neobhájil.
\cl
\freev
A dřív než tam
sebou seknu,
měl bych asi vymyslet,
jak tetě potom řeknu:
\cl
\repchorus{\emptyspace}
\num
Hej teto,
já ho fakt zabil,
ten bídák sežral magnet
a k topení se přitavil.
\fin
\freev
Hej teto,
co na to říci?
Můžeš si ho maximálně připnout na lednici.
\cl
\repchorus{\emptyspace}
\end{multicols}
\endsong