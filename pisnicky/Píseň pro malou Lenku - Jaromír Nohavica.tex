\beginsong{Píseň pro malou Lenku}[by={Jaromír Nohavica}]
%filter: Mlok
\begin{textblock}{5}(10,-0.5) \gtab{G/F\shrp{}}{200033:100034} \end{textblock}
\capo{2}
\num
\[Am\hidx{7}]{Jak mi tak} docházejí \[D\hidx{7}]síly, já \[G]pod jazykem \[G/F\shrp{}]cítím \[Em]síru
\[Am\hidx{7}]Víru, ztrácím \[D\hidx{7}]víru, a to mě \[G]míchá
\[Am\hidx{7}]Minomety reflektorů \[D\hidx{7}]střílí, jsem malým \[G]terčem \[G/F\shrp{}]{na bitevním} \[Em]poli
Kdekdo mě \[Am\hidx{7}]skolí, a to mě \[D\hidx{7}]bolí, u srdce \[G]píchá
\fin\chordsoff\num
Jak říká kamarád Pepa~-- co po mně chcete, slečno z první řady
Vaše vnady mě nebaví a trošku baví
Sudička moje byla slepá, když mi řekla to, co mi řekla
Píšou mi z pekla, že prý mě zdraví, že prý mě zdraví
\fin\chor
\chordson
\[Am\hidx{7}]Rána jsou smutnější než \[D\hidx{7}]večer, z rozbitého \[G]nosu \[G/F\shrp{}]{krev mi} \[Em]teče
Na čísle \[Am\hidx{7}]padesát šest jedna nula \[D\hidx{7}]devět nikdo to \[G]nebere
\[Am\hidx{7}]Rána jsou smutnější než \[D\hidx{7}]večer, na hrachu \[G]klečet, \[G/F\shrp{}]klečet, \[Em]klečet
\ldots \[Am\hidx{7}]Opustil mě můj děd \[D\hidx{7}]Vševěd a zas je \[G]úterek
\[Am\hidx{7}]{Jak se ti} \[D\hidx{7}]vede? No \[G]někdy fajn a \[G/F\shrp{}]někdy je to v \[Em]háji
Dva \[Am\hidx{7}]pozounisti z vesnické \[D\hidx{7}]kutálky pod okny mi \[G]hrají
\reppart{Tú tú \[Am\hidx{7}]{tú \ldots} \[D\hidx{7}] \[G] \[Em] \[Am\hidx{7}] \[D\hidx{7}] \[G]}
\chordsoff
\cl\num
Má malá Lenko, co teď děláš
Chápej, že čtyři roky, to jsou čtyři roky
A čas pádí a já jsem tady a ty zase jinde
Až nebudu, říkej, žes' mě měla
To pro tebe jsem skládal tyhle sloky
Na hrachu klečel a hloupě brečel a světu dával kvinde
\fin\chor
\ldots Opustil nás náš děd Vševěd a zas je úterek
Jak se vám vede? No někdy fajn a někdy je to v háji
Dva pozounisti z vesnické kutálky pod okny nám hrají
Tú tú tú \ldots
\cl
\endsong




