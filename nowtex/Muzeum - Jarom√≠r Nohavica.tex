\beginsong{Muzeum}[by={Jaromír Nohavica}]
\num
\[C]Ve Slezském muzeu \[G]{v oddělení} \[Am]třetihor
je bílý \[F]krokodýl a \[C]medvěd a liška a \[G\hidx{7}]kamenní \[C]trilobiti \[G\hidx{7}]
\[C]Chodí se tam jen tak, co \[G]noha nohu \[Am]mine
Abys viděl, jak ten \[F]život plyne
\[C]jaké je to všechno \[G\hidx{7}]pomíjivé \[C]živobytí \[G\hidx{7}]
\[F]Pak vyjdeš do parku a \[C\hidx{sus4}]{celou noc} se \[F]touláš noční Opavou
a \[B]opájíš se \[F]představou, jaké to \[C]bude \[F]{v ráji}
\fin\chor
V \[C]5:35 jednou z \[G]pravidelných \[Am]linek
Sedm zastávek do \[F]Kateřinek
\[C]Ukončete nástup, \[G\hidx{7}]dveře se \[C]zavírají
\cl\chordsoff\num
Budeš-li poslouchat a nebudeš-li odmlouvat
složíš-li svoje maturity
vychováš pár dětí a vyděláš dost peněz
Můžeš se za odměnu svézt na velkém kolotoči
Dostaneš krásnou knihu s věnováním~-- zaručeně
A ty bys chtěl plout na hřbetě krokodýla po řece Nil
A volat Tutanchamon~-- Vivat, vivat! po egyptském kraji
\fin\repchorus{\emptyspace}\num
Pionýrský šátek uvážeš si kolem krku
Ve Valtické poručíš si čtyři deci rumu a utopence k tomu
Na politém stole na ubruse píšeš svou rýmovanou Odysseu
nežli přijde někdo, abys šel už domů
Ale není žádné doma, jako není žádné venku
Není žádné venku, to jsou jenom slova
která obrátit se dle libosti dají
\fin\repchorus{\emptyspace}\num
Možná si k tobě někdo přisedne
A možná to zrovna bude muž
který osobně znal Egypťana Sinuheta
Dřevěnou nohou bude vyťukávat do podlahy
rytmus metronomu který tady klepe od počátku světa
Nebyli jsme, nebudem a nebyli jsme, nebudem
a co budem, až nebudem~-- jen navezená mrva v boží stáji
\fin\repchorus{\emptyspace}\num
Žena doma pláče a děti doma pláčí, pes potřebuje venčit
a stát potřebuje daň z přidané hodnoty
A ty si koupíš krejčovský metr a pak nůžkama odstříháváš
Pondělí, úterý, středy, čtvrtky, pátky, soboty
V neděli zajdeš do Slezského muzea
podívat se na vitrínu, kterou tam pro tebe už mají
\fin\repchorus{\emptyspace}
\endsong




