\beginsong{Marsyas a Apollón}[by={Marsyas}]
\num
\chordson
\[G]Ta krásná dívka, \[Am]{co se bojí} \[D]{o svoji krásu}
\[G]Athéna \[Em]jméno má, \[C]za starých \[G]dávných časů
\chordsoff
\fin\ifchorded\chordsoff\fi
\num
Odhodí flétnu, hrát nejde s nehybnou tváří
Kdo si ji najde dřív, tomu se přání zmaří
\fin\ifchorded\chordsoff\fi
\chor
\chordson
\[Em]Tak i Marsyas \[D]mámen flétnou \[Em]věří, že \[C]musí přetnout
[: \[G]Jedno pravidlo, \[D]sázku~-- a \[C]hrát \[D]líp než \[G]bůh :]
\chordsoff
\cl\ifchorded\chordsoff\fi
\num
Bláznivý nápad, snad nejvýš Marsyas míří
Apollón souhlasí, oba se s trestem smíří
Král Midas má říct, kdo je lepši, Apollón zpívá
O život soupeří, jen jeden vítěz bývá
\fin\ifchorded\chordsoff\fi
\chor
Tak si Marsyas mámen flétnou věří a musí přetnout
[: Jedno pravidlo, sázku~-- a hrát líp než bůh :]
\cl\ifchorded\chordsoff\fi
\num
Obrátí nástroj, už ví, že nebude chválen
Prohrál a zápolí podveden vůlí krále
Sám v tichém hloučku, sám na strom připraví ráhno
Satyra k hrůze všech zaživa z kůže stáhnou
\fin\ifchorded\chordsoff\fi
\chor
Tak si Apollón změřil síly~-- a každý se musel mýlit
[: Nikdo nemůže kouzlit a hrát líp než bůh :]
\cl\ifchorded\chordsoff\fi
\num
Ta krásná dívka, co se bála o svoji krásu
Dárkyně moudrostí za starých dávných časů
Teď v tichém hloučku, v jejích rukou úroda, spása
Athéna jméno má, chybí jí tvář a krása
\fin\ifchorded\chordsoff\fi
\chor
Ty dy dy\ldots
\cl\ifchorded\chordsoff\fi
\endsong



