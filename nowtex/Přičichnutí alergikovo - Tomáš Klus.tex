\beginsong{Přičichnutí alergikovo}[by={Tomáš Klus}]
\averse
\[Am]{V nose} mě šimrá \[Dm]alergie, \[F]jak se tak šourám \[E]alejemi
Ale \[Am]{je mi} fajn, \[Dm]dobře se tu žije, v mé \[F]rozkvéta\[E]jící \[Am]zemi
\cl
\chordsoff
\averse
Tak trochu slzím a pokašlávám, jak mi do očí padá pyl z květů
Nevěřím všem těm poplašným zprávám
Mně prostě a dobře je teď tu!
\cl
\averse
Pletence milenců v čerstvé trávě a křik dětí z pískoviště
Já věřím nejvíc své vlastní hlavě, mé tělo je mé utočiště
\cl
\bverse
\chordson
Tak mi \[F]doktoři \[E]pouštějí \[Am]žilou, to aby \[F]infekce \[E]nezpůso\[Am]bila
\chordsoff
Že bych snad nakazil taky svou milou zemi, co vskutku je čilá
\cl
\averse
Pověstná jarní apatie a dlouhé fronty před apatykou
Já a ty kouříme u Slávie, dál pak po nábřeží elektrikou
Kousek odtud je galerie, vrátného znám a za litr rumu
Uvidíš obrazy mistrů z Itálie a vrátného zlitého jak pumu
\cl
\averse
Rád se kochám a rád jsem kochán
Koho chleba jíš, toho píseň zpívej
Já zpívám tomu, koho zrovna potkám
Dělám všechno pro to, abych zůstal živej
\cl
\bverse
Páč mi doktoři dál pouštěj žilou
Infekce ve mně však už pokročila
Co nejde rozumem, musí jít silou, kampak se mrška skryla
\cl
\averse
Zápisy vznešených dialogů, citáty velkých mužů i žen
Kouřové signály úst demagogů
Zde domov můj~-- má rozkvetlá zem
Oteklá tvář, vstupenka do legií
Stíny tlouštíků na štíhlých křivkách cest
Partyzán napaden letardií, bezelstná vítězství slibů a gest
\cl
\averse
Na základní škole T. G. Masaryka
Naříká poslední učitel dějepisu
Znuděných studentů se výklad nedotýká
Budoucnost národa krapet mimo mísu
\cl
\bverse
Tak mi doktoři pustili žilou, infekce nátlaku ustoupila
Pak opustím taky svou milou zemi, čo zabudla, že mala Kryla
\cl
\endsong




