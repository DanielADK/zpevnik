\beginsong{Kometa}[by={Jaromír Nohavica}]
\num
\[Am]Spatřil jsem kometu, oblohou letěla
\chordsoff
Chtěl jsem jí zazpívat, ona mi zmizela
\chordson
\[Dm]Zmizela jako laň \[G\hidx{7}]{u lesa} v remízku
\[C]{V očích} mi zbylo jen \[E\hidx{7}]{pár žlutých} penízků
\fin
\chordsoff
\averse
Penízky ukryl jsem do hlíny pod dubem
Až příště přiletí, my už tu nebudem
My už tu nebudem, ach, pýcho marnivá
Spatřil jsem kometu, chtěl jsem jí zazpívat
\cl
\chor
\chordson
\[Am]{O vodě,} o trávě, \[Dm]{o lese}
\[G\hidx{7}]{O smrti,} se kterou smířit \[C]nejde se
\[Am]{O lásce,} o zradě, \[Dm]{o světě}
\[E]{A o} všech lidech, co \[E\hidx{7}]{kdy žili} na téhle \[Am]planetě
\cl
\num
Na hvězdném nádraží cinkají vagóny
Pan Kepler rozepsal nebeské zákony
Hledal, až nalezl v hvězdářských triedrech
Tajemství, která teď neseme na bedrech
Velká a odvěká tajemství přírody
Že jenom z člověka člověk se narodí
Že kořen s větvemi ve strom se spojuje
Krev našich nadějí vesmírem putuje
\fin
\chor
Na na na\ldots{}
\cl
\num
Spatřil jsem kometu, byla jak reliéf
Zpod rukou umělce, který už nežije
Šplhal jsem do nebe, chtěl jsem ji osahat
Marnost mne vysvlékla celého donaha
Jak socha Davida z bílého mramoru
Stál jsem a hleděl jsem, hleděl jsem nahoru
Až příště přiletí, ach, pýcho marnivá
Já už tu nebudu, ale jiný jí zazpívá
\fin
\chor
\ldots{} bude to písnička o nás a kometě
\cl
\endsong




