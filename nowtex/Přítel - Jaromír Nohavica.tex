\beginsong{Přítel}[by={Jaromír Nohavica}]
\begin{textblock}{5}(9,-0.5) \gtab{D/F\shrp{}}{200232:500132} \end{textblock}
\num
Jestlipak \[G]vzpomínáš si ještě na ten \[D/F\shrp{}]čas,
táhlo nám \[Em\hidx{7}]{na dvacet} a \[C]slunko \[G]bylo \[D]{v nás},
\[Am]vrabci nám jedli z ruky, \[C]život šel bez záruky,
\[G]ale taky bez pří\[D]kras.
\chordsoff
Možná, že hloupý, ale krásný byl náš svět,
zdál se nam opojný jak dvacka cigaret
a všechna tajná přání plnila se na počkání
anebo rovnou hned.
\fin\chor\chordson
\[Am]Kam jsme se poděli, \[C]kam jsme se to poděli
\[G]Kde je ti \[D/F\shrp{}]konec, můj \[Em]jediný příteli
\[Am]Zmizels mi nevím \[D\hidx{7}]kam
\[C\hidx{maj7}]{Sám, sám,} sám, zůstal jsem tu \[G]sám \[D/F\shrp{}] \[Em\hidx{7}] \[C] \[G] \[D]
\cl\chordsoff\num
Jestlipak vzpomínáš si ještě na tu noc,
jich bylo pět a tys mi přišel na pomoc.
Jó, tehdy nebýt tebe, tak z mých dvanácti žeber
nezůstalo příliš moc.
Dneska uz nevím, jestli přiběh by jsi zas,
jak tě tak slyším, máš už trochu vyšší hlas.
A vlasy, vlasy kratší, jó, bývali jsme mladší,
no a co, vem to ďas.
\fin
\repchorus{\ldots peru se teď sám}
\num
Jestlipak vzpomínáš si ještě na ten rok,
každá naše píseň měla nejmíň třicet slok
a my dva jako jeden ze starých reprobeden
přes moře jak přes potok.
Tvůj děda říkal: ono se to uklidní, 
měl pravdu, přišla potom spousta malých dní.
A byla velká voda, vzala nám, co jí kdo dal,
a tobě i to poslední.
\fin
\repchorus{\ldots zpívám tu teď sám}
\num
Jestlipak vzpomínáš si na to, jakýs' byl,
jenom mi netvrď, že tě život naučil,
člověk, to není páčka, kterou si kdo chce mačká,
to už jsem dávno pochopil.
A taky vím, že srdce rukou nechytím,
jak jsem se změnil já, tak změnil ses i ty,
a přesto líto je mi, že už nám nad písněmi
společný slunko nesvítí.
\fin
\repchorus{\ldots jsem tady sám}
\endsong



