\beginsong{Tři sudičky}[by={Jaromír Nohavica}]
\emptyv
\cseq{\[G] \[D] \[C] \[D] \[G]}\\
\cl
\chordsoff
\num
Tři sudičky u kolébky~-- hoj hoj~-- stály mi
Podle tvaru mojí lebky~-- hoj hoj~-- věštily
Já jsem zrovna dělal kaku~-- hoj hoj~-- do plének
Nechápal jsem smysl jejich~-- hoj hoj~-- myšlének
\fin
\chor
[: Hoj hoj hola hoj~-- budoucnost je boj
   Hoj hoj hola hoj~-- a tak se, chlapče, boj! :]
\cl
\num
První babka byla mírně~-- hoj hoj~-- obézní
Snad proto mi přivěštila ruku princezny
Ať s ní strávím ve svém loži~-- hoj hoj~-- co to dá
Čili jak to zpívá Samson~-- hoj hoj~-- pohoda
\fin
\repchorus{\emptyspace}
\num
Druhá babka byla zloduch~-- hoj hoj~-- od kosti
Snad proto mi přivěštila samé starosti
Jenom ať se hošík trochu~-- hoj hoj~-- potrápí
Ať je jeho princeznička~-- hoj hoj~-- na chlapy
\fin
\repchorus{\emptyspace}
\num
Třetí řekla~-- holky, vy snad~-- hoj hoj~-- blbnete
Víte, jaká nemoc vládne~-- hoj hoj~-- na světě
Ať zavládne v jeho žití~-- hoj hoj~-- idyla
Vzala nůž a fik!~-- a věštby~-- hoj hoj~-- zrušila
\fin
\repchorus{\emptyspace}
\num
Z toho plyne poučení~-- hoj hoj~-- pro všecky
Každý problém dá se řešit~-- hoj hoj~-- vědecky
Aby nekradlo se, uřež lidem~-- hoj hoj~-- ručičky
Aby nezdrhali za kopečky~-- hoj hoj~-- nožičky
\fin
\repchorus{\emptyspace}
\endsong




